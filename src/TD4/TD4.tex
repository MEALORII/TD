%% Do not edit unless you really know what you are doing.
\documentclass[french,english,12pt]{exam}

%\printanswers
\usepackage{tikz}
\usepackage{tikz}
\usepackage{../../latex/macro_mealor}
\usepackage{../../latex/structuralanalysis}
\usepackage[utf8]{inputenc}
\usepackage[T1]{fontenc} % accents codés dans la fonte
%\usepackage{layout}
\usepackage{a4wide}
\usepackage[frenchb]{babel}
\usepackage{hyperref}

\usepackage{graphicx}
\usepackage{caption}

\usepackage{newpxtext,newpxmath}
\usepackage{siunitx}

%\setlength{\hoffset}{0pt}
%\setlength{\oddsidemargin}{-1cm}   % Marge gauche sur pages impaires
%\setlength{\evensidemargin}{-1cm}   % Marge gauche sur pages paires
%\setlength{\marginparwidth}{0cm}   % Largeur de note dans la marge
%\setlength{\textwidth}{16cm}   % Largeur de la zone de texte (17cm)
\setlength{\voffset}{0pt}   % Bon pour DOS
\setlength{\marginparsep}{0pt}   % Séparation de la marge
\setlength{\topmargin}{0cm}   % Pas de marge en haut
\setlength{\headheight}{0cm}   % Haut de page
\setlength{\headsep}{0cm}   % Entre le haut de page et le texte
%\setlength{\footskip}{1cm}   % Bas de page + séparation
%\setlength{\textheight}{25.5cm}   % Hauteur de la zone de texte (25cm)

\usepackage{indentfirst}
 
\renewcommand{\solutiontitle}{\noindent\textbf{Solution :}\enspace}

\newcommand{\classurl}{\url{1}}
% #1 numéro de la feuille
% #2 titre de la feuille
\newcommand{\titre}[3] {%\textit{
  \begin{center}\textbf{\textsc{MEALOR II}}\\ \textit{Mécanique de l'endommagement et approche locale de la rupture}%\let\thefootnote\relax\footnotetext{\classurl} 
  \end{center}
 
  \noindent TD n\textdegree #1 \hfill  August 2023\\[-0.3cm]
  \rule{\linewidth}{.3mm}
  \vspace*{0.5pt}
  \begin{center}
    {
      \Large \bfseries { #2}
    }\\
    \vspace*{0.5cm}
	\large #3
    \vspace*{0.5cm}
  \end{center}
}
\usepackage{enumitem}
\usepackage{xcolor}
\definecolor{Blue}{RGB}{0,68,170}
\SolutionEmphasis{\normalfont\color{Blue}}
\DeclareCaptionFont{blue}{\color{Blue}}


\renewcommand{\thequestion}{\thesection.\arabic{question}}
%\patchcmd{\questions}{10.}{\thequestion.}{}{}% fix left margin

\newenvironment{objectifs}
    {\renewcommand{\labelitemi}{$\bullet$}\itshape\underline{Objectifs:}\begin{itemize}

    }
    { \itshape
    \end{itemize}\vspace{0.5em}
    }
    
\newcounter{Rfig}
\newenvironment{R_figure}
   {\begin{minipage}{\linewidth}\begin{center}\vspace{0.5mm}\stepcounter{Rfig}\addtocounter{figure}{-1}\renewcommand\thefigure{R-\arabic{Rfig}}
   \captionsetup{font=blue}}
   {\end{center}\vspace{0.5mm}\end{minipage}}
   
    
\newcounter{Rtab}
\newenvironment{R_table}
   {\begin{minipage}{\linewidth}\begin{center}\vspace{0.5mm}\stepcounter{Rtab}\addtocounter{table}{-1}\renewcommand\thetable{R-\arabic{Rtab}}
   \captionsetup{font=blue}
}
   {\end{center}\vspace{0.5mm}\end{minipage}}
   
      
\graphicspath{{./pic/}}
\usepackage{subcaption}
\begin{document}
\thispagestyle{empty}
\titre{2}{Les modèles variationnelles de rupture en 1d: barre en traction}{Jérémy Bleyer, Corrado Maurini}
%\maketitle
\begin{objectifs}
\item Manipuler les conditions d'optimalité d'une fonctionnelle et appliquer l'approche variationnelle à la rupture dans le cas 1d
\item Déterminer l'évolution quasi-statique d'un modèle d'endommagement
\item Comprendre l'origine des contraintes limites et de la ténacité dans les modèles d'endommagement à gradient.
\item Assimiler les bases des modèles de rupture de type champs de phase avant la mise en ouvre numérique.
\item Construire des solutions de référence pour la vérification d'un code numérique de type champs de phase.
\end{objectifs}

\section*{Contexte}
Les modèles variationnelle à rupture sont à la base des approches de type champs de phase pour la simulation numérique des phénomènes de rupture fragile. 
Dans cet exercice on se propose de retrouver les propriétés fondamentales de ces modèles dans le cadre simplifié d'une barre en traction. 


\begin{figure}[h!]
\begin{center}
    \begin{tikzpicture}
    \point{a}{0}{0};
    \point{b}{6}{0};
    \point{e}{6.1}{0};
    \point{d}{7.5}{0};
    \point{c}{3}{0};
    \support{3}{a}[-90];
    %\support{3}{b}[90];
    \beam{2}{a}{b}[0][1];
    \dimensioning{3}{e}{d}{.0}[$u(L)=t\,L$];
    %\notation{1}{c}{$\epsilon_0(t)=-t<0$}[above];
    \dimensioning{1}{a}{b}{-1.}[$L$];
    %\load{1}{d}[180]
    %\notation{3}{a}{b}[$i$];
    %\temperature{a}{b}{-.5}{.5};
    \end{tikzpicture}
\end{center}
    \caption{Barre en traction avec déplacement imposé à l'extrémité.}\label{fig:barre}
\end{figure}
On considère une barre encastrée en $x=0$ et avec un déplacement imposé $u(L)=t\,L$ en  $x=L$.
$x=L$.  On note par $E_0$ le module de Young du matériau
constituant la barre (dans son état non endommagé), $G_c$ sa
ténacité, $S$ la
surface de la section droite et $L$ la longueur. Dans la suite on étudie le
problème de rupture de la barre avec un modèle de Griffith et un modèle
d'endommagement à gradient.

\section{Solution élastique : approche variationnelle}
On suppose ici que la barre soit élastique linéaire. On se propose de déterminer la solution de ce problème simple avec l'approche variationnelle de l'élasticité.

L'énergie potentielle élastique de la barre est 
\begin{equation}
	\mathcal P(u)=\int_\Omega \frac{E_0S}{2}\left(
	\frac{du}{dx}
		(x)\right)^2 dx
	\label{eq:energie-elastique}
\end{equation}
où $u:x\in\Omega\equiv[0,L]\to u(x)\in\mathcal R$   le déplacement axial de la
barre.
Le problème de recherche de configurations d'équilibre peut être formulée comme problème de minimisation de l'énergie. La configuration d'équilibre $u$ doit être celle qui minimise l'énergie parmi toutes les configurations admissibles $\hat u \in\mathcal{C}_t$, où $\mathcal{C}_t$ est l'espace des configurations admissibles :
\begin{equation}
u\in\mathcal{C}_t: \,\mathcal{P}(u+h(\hat u-u))-\mathcal
{P}(u)\geq 0,\forall \hat u\in\mathcal C_t.
\label{eq:energy-variation}
\end{equation}
L'éspace de configuration admissibles est donné par les fonctions qui respectent les conditions aux limites de Dirichlet et qui sont assez régulières pour que $\mathcal{P}(u)\leq +\infty$:
\begin{questions}
    \question 
    Donner l'expression de $\mathcal{C}_t$ pour la barre de figure~\ref{fig:barre} et monter, avec une développement limité en $h$ de la variation de l'énergie~\eqref{eq:energy-variation} que une configuration d'équilibre doit satisfaire la condition suivante, formulation faible du problème d'élasticité:
    \begin{equation}
        D_u\mathcal{P}(u)(v)=0,\quad \forall v\in \mathcal{C}_0
        \label{eq:elasticite-foc}
    \end{equation}
    où on note 
    $$
    D_u\mathcal{P}(u)(\hat u):=\left.\dfrac{d}{dh}\mathcal{P}(u+hv)\right\vert_{h=0}
    $$
    et donner pour notre exemple l'expression de 
    $\mathcal{C}_0$, définie comme l'éspace vectorielle des variations admissibles, constituées par les différences entre fonctions admissibles.
    \question Calculer $D_u\mathcal{P}(u)(v)$ et montrer que la condition~\eqref{eq:elasticite-foc} implique la condition d'équilibre
    \begin{equation}
        \frac{d}{dx}\sigma(x)=0,\quad \text{avec}\quad\sigma=E_0S\frac{d u}{dx}(x).
        \label{eq:equilibre}
    \end{equation}
    \question Déterminer les champs de déplacement $u_t$ et contrainte $\sigma_t$ solution du problème.  Montrer que l'énergie potentielle à l'équilibre en fonction du chargement est :
    \begin{equation}
        \mathcal{P}(u_t)=\min_{u\in\mathcal{C}_t}\mathcal{P}(u)=E_0S\,L\,\frac{t^2}{2}
        \label{eq:energie-elastique-sol}
    \end{equation} 
    \begin{solution}
        L'équation d'équilibre donne :
    $$
    E_0S\left(\frac{du_t}{dx}(x)\right)=\sigma_t\Rightarrow \frac{du_t}{dx}(x)=\frac{\sigma_t}{E_0S}\Rightarrow u_t =\left(\frac{\sigma_t}{E_0S}\right)x+u(0)
    $$
    En utilisant les conditions aux limites $u\in\mathcal C_t\rightarrow u(0)=0,u(L)=t\,L$ on trouve l'unique solution
    $$
    u_t=t x,\quad \sigma_t=E_0S\,t
    $$
    En remplaçant dans l'énergie potentielle~\eqref{eq:energie-elastique} on trouve~\eqref{eq:energie-elastique-sol}.
    \end{solution}
\end{questions}

\section{Modèle de Griffith}
Dans un modèle de type "interphase franche" de rupture à la Griffith, les fissures un ensemble de points $\Gamma\equiv\{\bar x_i\}_{i=1}^n$ où les déplacements peuvent être discontinus.  
On considère ici pour la barre un modèle de Griffith monodimensionnel,
caractérisée par l'énergie totale
\begin{equation}
	\quad \mathcal{E}_G(u,n)=\int_{\Omega\setminus\Gamma} E_0S\left(
	\frac{du}{dx}
		(x)\right)^2 dx + G_c\,S \,n
	\label{minprob}
\end{equation}
où $u:x\in[0,L]\to u(x)\in\mathcal R$ représente  le déplacement axial de la
barre et $n$ le nombre des fissures. 
L'approche variationnelle de la rupture (Francfort-Marigo 1998) défini la solution du problème comme le minimum globale de l'énergie $\mathcal{E}_G(u,n)$, sous une contrainte d'irréversibilité pour les fissures.


\begin{questions}
\question En utilisant les résultats de la section précedente, calculer $\mathcal{E}_G(u,0)$ et $\mathcal{E}_G(u,1)$ en fonction de $t$. Calculer le chargement critique $t_G$ pour la fissuration de la barre selon ce modèle. Discuter de la pertinence physique de ce modèle de nucléation. 
\begin{solution}
    $$\mathcal{E}_G(u,0)=\dfrac{t^2}{2}{E_0 S}L,\quad \mathcal{E}_G(u,1)=G_c S
    \quad\Rightarrow\quad t_G=\sqrt{\dfrac{2G_c}{E_0SL}}$$
    La dépendance de la charge critique par rapport à la longueur de la barre n'est pas raisonnable. On a un chargement critique infini (et contraintes infinies) pour $L\to 0$ et nul pour $L\to\infty$. 
\end{solution}

\end{questions}


\section{Modèle d'endommagement à gradient}
%\begin{figure}[htbp]
%	\begin{center}
%		\includegraphics[width=.4\textwidth]{barre1D}
%\caption{default}
%		\label{default}
%	\end{center}
%\end{figure}
On étudie le problème d'évolution de la barre en figure~\ref{fig:barre} avec un modèle d'endommagement à gradient en 1d et une approche énergétique.  On considère le problème d'évolution à temps discret, avec $t_i=i\,\Delta t$ et un pas $\Delta t>0$ constant. Soit $u_i:x\in[0,L]\to\mathcal R$, $\alpha_i:x\in[0,L]\to\mathcal R$ le champ de déplacement axial et d'endommagement à l'instant $t_i$.
En $t=0$ la barre est à endommagement nul, $\alpha_0=0$. 
On suppose le champs d'endommagement libre aux bords. A l'instant $t_i=
i\,\Delta t$, l'état $(u_i,\alpha_i)$ de la barre est donné par la solution du
problème suivaint de minimisation locale directionnelle :

Déterminer $u_i\in \mathcal C_i,\;\alpha_i\in\mathcal{D}(\alpha_{i-1})$ tels que 
$\forall \hat u\in\mathcal{C}_i$,
$\forall \hat \alpha\in\mathcal{C}(\alpha_{i-1})$,
$\exists \bar h\geq 0$: 
\begin{equation}
    \boxed{
	\mathcal{E}(u_i+h(\hat u-u_i),{\alpha_i+h(\hat \alpha-\alpha_i)})-\mathcal{E}(u_i,{\alpha_i})\geq 0,\quad \forall h\in(0, \bar h)
	\label{minprob}
    }
\end{equation}
où
\begin{equation}
	\mathcal{E}_{\ell}(u,{\alpha})=\left(
	\int_{0}^L \dfrac{E_0}{2}  \, a({\alpha(x)})\,\left(\dfrac{du}{dx}
		(x)\right)^2+
	\,w_1 \,\left({w({\alpha})}+{\ell^2}\, {\left(\dfrac{d\alpha}
						{dx}
						(x)\right)^2}\right)\right)dx,
						\label{eq:energy}
\end{equation}
%et $\mathcal{U}_i\equiv\{u: u(0)=0,\; u(L)=0\} $,  $\mathcal{D}_i\equiv
% \{\alpha: \alpha\geq\alpha_{i-1}\;\forall x\in[0,L]\} $,
$E_0$ étant le module de Young de la barre saine, $\ell$ la longueur interne.
La constante $w_1$ représente l'énergie dissipée pour unité de volume dans le
modèle d'endommagement.
On considère dans la suite le modèle d'endommagement caractérisé par
\begin{equation}
	a(\alpha)=(1-\alpha)^2,\qquad w(\alpha)=\alpha.
	%	a(\alpha)=(1-\alpha)^2,\qquad w(\alpha)=\alpha^2
	\label{aw}
\end{equation}

\subsection{Formulation faible}
\begin{questions}
\question Montrez que toute solution du problème de minimisation doit satisfaire ces conditions d'optimalité du premier ordre par rapport à $u$ et $\alpha$:
\begin{eqnarray}
    D_u\mathcal{E}(u_i,{\alpha_i})(v)&=&0,\quad \forall v\in\mathcal{C}_0
    \label{eq:foc-u}\\
    D_\alpha\mathcal{E}(u_i,{\alpha_i})(\hat \alpha-\alpha_{i})
    &\geq &0,\quad\forall  \hat \alpha\in \mathcal{D}(\alpha_{i})
    \label{eq:foc-v}
\end{eqnarray}
Ce système constitue la formulation faible en espace du problème et est à la base de l'implémentation numérique proposée en TP. 
Il s'agit d'un système couplé d'une équation variationnelle pour $u$ (équilibre) et une inéquation variationnelle pour $\alpha$ (critère d'endommagement). 
\question
Calculer les dérivées directionnelles $D_u\mathcal{E}(u_i,{\alpha_i})(v)$ et  $D_\alpha\mathcal{E}(u_i,{\alpha_i})(\beta)$.
\end{questions}
\subsection{Formulation forte}

\begin{questions}
    \question 
    En reprenant les raisonnements de la section 1, on peut montrer que la condition d'optimalité du premier ordre par rapport à $u$ est équivalente aux conditions d'équilibre suivantes :
\begin{equation}
    \dfrac{d\sigma}{dx}(x) = 0, \; \forall x\in(0,L),\quad \text{avec}\quad
    \sigma = E_0 \,a(\alpha(x))  \left(\dfrac{du}{dx}(x)+t\right), \
    \label{eq:equilibre}
\end{equation}
où $\sigma$, indépendant de $x$, représente la valeur de la contrainte dans la barre.\\

    \question Montrez  que, en supposant les solutions assez régulières, la condition d'optimalité du premier ordre par rapport à $\alpha$ est vérifiée si et seulement si, pour $x\in(0,L)$ implique que 
    \begin{equation}
        \dfrac{E_0\,a'(\alpha_i)}{2}\left(\frac{du_i}{dx}\right)^2+\left(w'(\alpha_i) -2 \ell^2\frac{d^2\alpha_i}{dx^2}\right)\,w_1 \geq{0}
    \end{equation}
    et qu'où $\alpha_i >\alpha_{i-1}$ la condition ci-dessus doit être satisfaite comme une égalité. On conclut que les conditions d'optimalité par rapport à $\alpha$ ont la suivante écriture en forme forte:
\begin{subequations}
    \begin{align}
    \alpha_i-\alpha_{i-1}&\geq0,\\
    \dfrac{E_0\,a'(\alpha_i)}{2}\left(\frac{du_i}{dx}\right)^2+\left(w'(\alpha_i) -2 \ell^2\frac{d^2\alpha_i}{dx^2}\right)\,w_1 &\geq{0},\\
    \left(
      \dfrac{E_0\,a'(\alpha_i)}{2}\left(\frac{du_i}{dx}\right)^2+\left(w'(\alpha_i) -2 \ell^2\frac{d^2\alpha_i}{dx^2}\right)\,w_1\right)\,
    (\alpha_i-\alpha_{i-1})&=0,
    %&\quad\text{on }&\Omega,
    %\\
    %\alpha-\alpha_{i-1}&\geq 0,
    %&
    %\nabla{\alpha}\cdot{\vect{n}}&\geq {0},
    %&
    %(\nabla{ \alpha}\cdot{\vect{n}})(\alpha-\alpha_{i-1})&=0
    %&\text{on }& \partial\Omega,
    \end{align}
  \end{subequations}
  Ces conditions sont complétés par les conditions aux bords
  \begin{subequations}
    \begin{eqnarray}
        \alpha_i(0)- \alpha_{i-1}(0)\geq 0,&\alpha_i'(0)\leq 0,& (\alpha_i(0)- \alpha_{i-1}(0) )(\alpha_i'(0))=0,\\
        \alpha_i(L)- \alpha_{i-1}(L)\geq 0,&\alpha_i'(L)\geq 0,& (\alpha_i(L)- \alpha_{i-1}(L) )(\alpha_i'(L))=0.
    \end{eqnarray}
    \label{eq:bc-alpha}
\end{subequations}
\end{questions}
\subsection{Solution homogène}
On cherche à présent les solutions à endommagement homogène en espace, en fonction de $t$. 
\begin{questions}
  \question Montrer que le critère d'endommagement peut se réécrire sous la forme:
  \begin{equation}
   \vert\sigma\vert\leq\bar{\sigma}(\alpha):= \sqrt{\frac{2 E_0 w_1w'(\alpha)}{s'(\alpha)}}  
\end{equation}
  où $s(\alpha)=1/a(\alpha)$ est la modulation de la souplesse.
    %\item Montrer que si l'endommagement est homogène et inférieur à
%          1, le déplacement est nul en tout point ($u=0$).
    \question Pour les modèle $\mathsf{AT_1}$ et $\mathsf{AT_2}$, determiner la contrainte maximale supportable 
     $$\sigma_M=\max_{\alpha\in[0,1]}\bar\sigma(\alpha)
     $$
     et les valeurs de l'endommagement pour lesquels les modèles est durcissant/adoucissant en contraintes (\emph{stress hardening/softening}).
    
  \begin{solution}
      Le critère pour $d\alpha/dx=0$ donne:
      $$
      s(\alpha)=\frac{1}{(1-\alpha)^2},\quad s'(\alpha)=\frac{1}{(1-\alpha)^3}, \quad s''(\alpha)=\frac{3}{(1-\alpha)^4}
      $$
  \end{solution}
    \question Pour un chargement monotone croissant $t$ à partir d'un état initiale $\alpha_0=0$, donner l'évolution de l'endommagement et de la contrainte $\sigma$ avec $t$ pour les modèles $\mathsf{AT_1}$ et $\mathsf{AT_2}$. 
    Montrez que les réponses $\sigma-t$ sont cohérentes avec les réponses ci-dessous. 
    {%\begin{figure}[htbp]
     \centering
      \includegraphics[width=0.45\textwidth]{at1hom}
      \includegraphics[width=0.45\textwidth]{at2hom}\\
    %  \caption{Réponse en contrainte $\sigma$ en fonction du chargement $t$ monotone croissant pour les modèles $\mathsf{AT_1}$ et $\mathsf{AT_2}$.}
    %  \label{fig:response-homogene}
    %\end{figure}
     }
    Marquez dans ces graphes les valueurs de contraintes maximales $\sigma_M$, de la contrainte limite d'élasticité $\sigma_e$ et des valeurs correspondantes de $t$. Marquez les zones à régime durcissant/adoucissant en contraintes. 
\end{questions}

\subsection{Solutions localisées}
On cherche à présent les solutions à endommagement non-homogène. En particulier on cherche des solutions localisées dans une bande de largeur $D$ inconnue à l'intérieur du domaine.
$$
\begin{cases}
\alpha(x)>0, &x\in  (x_0-D/2,x_0+D/2),\\
\alpha(x)=0, & \text{autrement}
\end{cases}
$$
avec $x_0$ et $D$ tels que $0<x_0\pm D/2<L$.
Les solutions de ce type pour lesquels  $\max_x\alpha(x)=1$,  donc $\sigma=0$, sont assimilables à une approximation diffuse des fissures dans le modèle d'endommagement. 
On peut donc identifier la ténacité  effective (énergie dissipée pour la création d'une fissure)  du modèle d'endommagement avec l'énergie dissipée dans ce type de solutions :
$$
\tilde G_c=w_1\int_{x_0-D/2}^{x_0+D/2}w(\alpha(x))+\ell^2\left(\frac{d\alpha}{dx}\right)^2\,dx
$$
	 \begin{questions}
		      %\item \emph{Intégrale première}.  Montrer que, étant $c$ une constante,
		      %$$-s(\alpha) \dfrac{E_0\sigma^2}{2}+\dfrac{G_c}{c_w}\left(\dfrac{w(\alpha)}{\ell} - 2 \ell \alpha'^2\right) = c\qquad \forall x\in(0,L)
		      %$$
		      %avec
		      %$c = - {E_0\sigma^2}/{2}$.
    \question Montrez que les solutions doivent vérifier l'intégrale première suivante :
              $$
              \frac{E_0}{w_1}(s(\alpha)-s(0))\,{\sigma^2}+2\,w(\alpha)=\left(\frac{d\alpha}{dx}\right)^2\ell^2
              $$
    \question Montrez que 
              $$
              \tilde G_c=c_w{w_1\ell},\text{ avec }c_w=4\,\int_0^1\sqrt{w
              (\alpha)}d\alpha.
              $$
              et calculer $c_w$ pour les modèles $\mathsf{AT}_1$ et $\mathsf{AT}_2$.
	\question On cherche des solutions où la contrainte est nulle, avec $\sigma=
             0$ et
              $\alpha(x_0)=1$. %, voir Figure~\ref{atkhom}(droite). 
              On suppose
              aussi que $\alpha'(x_0\pm D/2)=0$.
              Déterminer le profil d'endommagement respectant ces conditions pour le modèle $\mathsf{AT}_1$.
		            

	\question Donner la valeur de la constante $w_1$ à utiliser pour que
	      l'énergie dissipée dans ces solutions localisées avec $\sigma=0$ soit
	      équivalente à $G_c$, i.e.~à l'énergie dissipée dans une fissure dans le
	      modèle de rupture de Griffith.
%	      \newline\underline{\textbf{Réponse 7:}}\vskip 5cm
	      %$$
	      %c_w = 4 \int_0^1\sqrt{w(\alpha)}d\alpha  = 8/3
	      %$$
	      %et  que la limite d'élasticité de la barre est donnée par
	      %$$\sigma_c= \sqrt{\dfrac{3\,G_cE_0}{8\,\ell}}$$
	\question  Étant données les propriétés matérielles du béton: $E_0=30\,
    \mathrm{GPa}$, $G_c =50 \mathrm{N/m}$,  $\sigma_c = 4 \mathrm{MPa}$
    déterminer $\ell$.
 %   \newline\underline{\textbf{Réponse 8:}}\vskip 5cm

    \end{questions}


\end{document}
