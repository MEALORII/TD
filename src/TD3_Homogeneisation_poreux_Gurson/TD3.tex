\documentclass[french,12pt]{exam}

%\printanswers

\usepackage{../../latex/macro_mealor}
\usepackage[utf8]{inputenc}
\usepackage[T1]{fontenc} % accents codés dans la fonte
%\usepackage{layout}
\usepackage{a4wide}
\usepackage[frenchb]{babel}
\usepackage{hyperref}

\usepackage{graphicx}
\usepackage{caption}

\usepackage{newpxtext,newpxmath}
\usepackage{siunitx}

%\setlength{\hoffset}{0pt}
%\setlength{\oddsidemargin}{-1cm}   % Marge gauche sur pages impaires
%\setlength{\evensidemargin}{-1cm}   % Marge gauche sur pages paires
%\setlength{\marginparwidth}{0cm}   % Largeur de note dans la marge
%\setlength{\textwidth}{16cm}   % Largeur de la zone de texte (17cm)
\setlength{\voffset}{0pt}   % Bon pour DOS
\setlength{\marginparsep}{0pt}   % Séparation de la marge
\setlength{\topmargin}{0cm}   % Pas de marge en haut
\setlength{\headheight}{0cm}   % Haut de page
\setlength{\headsep}{0cm}   % Entre le haut de page et le texte
%\setlength{\footskip}{1cm}   % Bas de page + séparation
%\setlength{\textheight}{25.5cm}   % Hauteur de la zone de texte (25cm)

\usepackage{indentfirst}
 
\renewcommand{\solutiontitle}{\noindent\textbf{Solution :}\enspace}

\newcommand{\classurl}{\url{1}}
% #1 numéro de la feuille
% #2 titre de la feuille
\newcommand{\titre}[3] {%\textit{
  \begin{center}\textbf{\textsc{MEALOR II}}\\ \textit{Mécanique de l'endommagement et approche locale de la rupture}%\let\thefootnote\relax\footnotetext{\classurl} 
  \end{center}
 
  \noindent TD n\textdegree #1 \hfill  August 2023\\[-0.3cm]
  \rule{\linewidth}{.3mm}
  \vspace*{0.5pt}
  \begin{center}
    {
      \Large \bfseries { #2}
    }\\
    \vspace*{0.5cm}
	\large #3
    \vspace*{0.5cm}
  \end{center}
}
\usepackage{enumitem}
\usepackage{xcolor}
\definecolor{Blue}{RGB}{0,68,170}
\SolutionEmphasis{\normalfont\color{Blue}}
\DeclareCaptionFont{blue}{\color{Blue}}


\renewcommand{\thequestion}{\thesection.\arabic{question}}
%\patchcmd{\questions}{10.}{\thequestion.}{}{}% fix left margin

\newenvironment{objectifs}
    {\renewcommand{\labelitemi}{$\bullet$}\itshape\underline{Objectifs:}\begin{itemize}

    }
    { \itshape
    \end{itemize}\vspace{0.5em}
    }
    
\newcounter{Rfig}
\newenvironment{R_figure}
   {\begin{minipage}{\linewidth}\begin{center}\vspace{0.5mm}\stepcounter{Rfig}\addtocounter{figure}{-1}\renewcommand\thefigure{R-\arabic{Rfig}}
   \captionsetup{font=blue}}
   {\end{center}\vspace{0.5mm}\end{minipage}}
   
    
\newcounter{Rtab}
\newenvironment{R_table}
   {\begin{minipage}{\linewidth}\begin{center}\vspace{0.5mm}\stepcounter{Rtab}\addtocounter{table}{-1}\renewcommand\thetable{R-\arabic{Rtab}}
   \captionsetup{font=blue}
}
   {\end{center}\vspace{0.5mm}\end{minipage}}
   
      
\graphicspath{{./pic/}}
\usepackage{float}

\newcommand{\todo}[1]{\textbf{\textcolor{red}{[ToDo: #1]}}}
\begin{document}
\thispagestyle{empty}
\titre{3}{Rupture ductile : homogénéisation de matériaux poreux}{Jérémy Hure, Djimédo Kondo, Jérémy Bleyer}

\begin{objectifs}
\item Obtenir un critère de plasticité pour matériaux poreux dans le cadre de la modélisation de la rupture ductile
\item Analyser ce critère et dériver l'évolution de la porosité
\item Discuter des effets d'anisotropie
\end{objectifs}

\section{Position du problème}
Le point de départ est de considérer le \textbf{critère de plasticité de von Mises} qui s'écrit:
\begin{equation}
  \phi\left( \td{\sigma}   \right) =  \sigma_{eq} - \sigma_0 \leq 0 \hspace{2cm} \sigma_{eq} = \sqrt{\frac{3}{2} \td{s} \cdot \td{s}} \ \ \ \ \mathrm{avec} \ \ \ \ \td{s} = \tsigma - \frac{1}{3} \mathrm{tr}\tsigma \td{I}
  \label{eq1}
\end{equation}
La déformation plastique se produit quand le critère est atteint, \textit{i.e.}, $\phi = 0$. Dans la suite, l'écrouissage n'est pas pris en compte ($\sigma_0 = \text{cste}$) et l'élasticité est ignorée, ce qui correspond à un matériau dit \textit{rigide parfaitement plastique}.

La loi d'écoulement plastique associée permet d'obtenir le taux de déformation plastique:
\begin{equation}
  \td{d} = d_{eq} \frac{3 \td{s}}{2 \sigma_0}  \hspace{2cm} d_{eq} = \sqrt{\frac{2}{3}\td{d} \cdot \td{d} }
  \label{eq200}
\end{equation}

Afin d'étendre ce critère à la présence d'une porosité, la géométrie présentée sur la Fig.~\ref{fig1} est considérée, à savoir un cylindre creux de hauteur $L$, de rayons intérieur $a$ et extérieur $b$. La porosité, définie comme le rapport entre le volume de la cavité et le volume total est noté $f = (a/b)^2$.
\begin{figure}[H]
  \centering
  \includegraphics[height = 7cm]{homoMEALOR.pdf}
\caption{Géométrie considérée et critère de plasticité}
\label{fig1}
\end{figure}

La contrainte et le taux de déformation (plastique) macroscopique sont définis à l'échelle du cylindre par les relations suivantes:
\begin{align}
  \tSigma &= \frac{1}{V} \int_V \tsigma dV\\
  \td{D} &= \frac{1}{V} \int_V \td{d} \, dV \\
  &= \frac{1}{2V} \int_{\partial V} \left(\td{v} \otimes \td{n} + \td{n} \otimes \td{v}     \right) dS \label{eq:D-v-n}
\end{align}
avec $\td{v}$ le champ de vitesse, $V$ le cylindre de rayon $b$ et $\td{n}$ la normale à la surface extérieure.\\


\begin{questions}
\question Quel matériau peut être représenté par cette géométrie ? Dans le cas de chargement axisymétrique, montrer que le critère de plasticité en l'absence de cavité est donné par le critère de von Mises de la matrice et correspond à la Fig.~\ref{fig1} où $\Sigma_\ell$ désigne la contrainte latérale dans le plan $(x,y)$.\\
\end{questions}

\begin{solution}
Le fait de considérer des pores cylindriques correspond à une idéalisation de cavités très allongées dans la direction $z$. Ce type de situation peut se recontrer dans le cas de tôles d'aciers laminés à chaud.\\

En l'absence de cavité, le champ de contrainte est homogène et égal au champ macroscopique:
$$\td{\sigma} = \Sigma_{xx}\td{e}_x\otimes\td{e}_x + \Sigma_{yy}\td{e}_y\otimes\td{e}_y +  \Sigma_{zz}\td{e}_z\otimes\td{e}_z$$
avec $\Sigma_{xx}=\Sigma_{yy}=\Sigma_\ell$ pour un chargement axisymétrique. Dans ce cas, le déviateur est:
$$\td{s} = \dfrac{1}{3}(\Sigma_{\ell}-\Sigma_{zz})(\td{e}_x\otimes\td{e}_x + \td{e}_y\otimes\td{e}_y - 2\td{e}_z\otimes\td{e}_z)$$
De sorte que le critère s'exprime comme:
$$\sigma_{eq} = |\Sigma_\ell - \Sigma_{zz}| \leq \sigma_0$$
qui correspond bien au cylindre d'axe $\Sigma_{\ell}$ de la figure.
\end{solution}


Nous cherchons à obtenir le critère de plasticité du cylindre, c'est-à-dire l'équivalent de l'Eq.~\ref{eq1} pour les grandeurs macroscopiques $\tSigma$ et $\td{D}$ en présence d'une cavité. Pour cela, la théorie de l'analyse limite est utilisée. En pratique, celle-ci consiste à postuler un champ de vitesse $\td{v}$ admissible (dans un sens qui sera détaillé dans la suite), et à calculer la dissipation plastique associée:
\begin{equation}
  \Pi = \frac{1}{V} \int_V \sigma_0 d_{eq}(\td{v}) \,dV
  \label{eq3}
\end{equation}

Sous certaines conditions (qui là-encore seront détaillées dans la suite), cette dissipation plastique permet d'évaluer la contrainte maximale admissible, c'est-à-dire le critère de plasticité macroscopique, par l'équivalence suivente:
\begin{equation}
  \tSigma \cdot \td{D} \leq \Pi(\td{D}) \quad \forall \td{D} \quad \Longleftrightarrow \quad \Phi(\tSigma) \leq 0 \label{eq2}
\end{equation}
où $\Phi$ désigne le critère de plasticité macroscopique. De plus, on peut montrer que la frontière $\Phi(\tSigma^*)=0$ du critère est décrite par les états de contrainte tels que:
\begin{equation}
\tSigma^*\cdot\td{D} = \Pi(\td{D}) \quad \Longleftrightarrow \quad \tSigma^* = \dfrac{\partial \Pi}{\partial \td{D}}
  \label{eq:plastic-boundary}
\end{equation}
La frontière est ainsi paramétrée par la valeur\footnote{En réalité, l'expression \eqref{eq:plastic-boundary} ne dépend pas de la norme de $\td{D}$ mais uniquement de sa direction} de $\td{D}$.

\section{Champ de vitesse virtuel et dissipation plastique}

Afin d'obtenir une estimation du critère de plasticité, la première étape consiste donc à choisir un champ de vitesse pris sous la forme:
\begin{equation}
  \td{v} = v_r(r) \td{e}_r + v_z(z) \td{e}_z \label{eq:champ-vitesse}
\end{equation}

\begin{questions}
\question En imposant que le champ de vitesse soit incompressible, montrer que celui-ci est de la forme:\\
\begin{equation}
  \td{v} = \left( \frac{B}{r} -  \frac{A}{2}r \right)  \td{e}_r +  A z \td{e}_z 
\end{equation}
\begin{solution}
L'incompressibilité s'écrit ici:
$$\mathrm{tr}\td{d} = \divergence{\td{v}} = v_r'(r)+\dfrac{v_r(r)}{r}+v_z'(z)=0$$
Comme les deux premiers termes ne dépendent que de $r$ et le troisième que de $z$, ce dernier est nécessairement constant, c'est-à-dire: $v_z(z) = Az$. De plus, on a ainsi:
$$v_r'(r)+\dfrac{v_r(r)}{r}= \dfrac{1}{r}(rv_r)'=-A$$
Ce qui donne après intégration:
$$rv_r(r) = B -  \frac{A}{2}r^2$$
soit au final:
\begin{equation}
  \td{v} = \left( \frac{B}{r} -  \frac{A}{2}r \right)  \td{e}_r +  A z \td{e}_z 
\end{equation}
où $A$ et $B$ sont deux constantes indéterminées.
\end{solution}

\question Comment s'écrit le taux de déformation macroscopique $\td{D}$ ? \\
\begin{solution}
Le taux de déformation microscopique vaut:
$$\td{d} = \left(-\frac{A}{2}-\frac{B}{r^2}\right)\td{e}_r\otimes\td{e}_r + \left(-\frac{A}{2}+\frac{B}{r^2}\right)\td{e}_\theta\otimes\td{e}_\theta + A\td{e}_z\otimes\td{e}_z$$
On constate donc que le terme en $A$ est homogène et est donc associé à un tenseur de déformation macroscopique de la forme $-\frac{A}{2}\td{1}_p + A\td{e}_z\otimes\td{e}_z$ où $\td{1}_p$ représente l'identité du plan $(\td{e}_x,\td{e}_y)$. 

Le deuxième terme en $B$ est associé au champ de vitesse purement radial $\frac{B}{r}\td{e}_r$. Il est donc associé au tenseur de déformation donné par:
\begin{align*}
\frac{1}{2V} \int_{\partial V} \left(\td{v} \otimes \td{n} + \td{n} \otimes \td{v}     \right) dS &= \frac{1}{V}\int_{\partial V} \dfrac{B}{b} \td{e}_r\otimes\td{e}_r dS \\
&= \dfrac{L B}{\pi b^2 L} \int_0^{2\pi}\td{e}_r\otimes\td{e}_r d\theta\\
&= \dfrac{B}{b^2} \td{1}_p
\end{align*}
\textbf{Attention!} on ne peut utiliser directement l'intégrale de volume entre $a$ et $b$ car le champ ne peut pas être prolongé dans le pore n'importe comment, en particulier pas par $\td{d}=0$. Il est donc plus sûr d'utiliser la relation \eqref{eq:D-v-n} qui ne dépend pas de la façon dont on prolonge le champ dans le bord, cf. Leblond, J. B. "Mécanique de la rupture fragile et ductile, 2000.

On a donc au final:
$$\td{D} = \left(\dfrac{B}{b^2}- \dfrac{A}{2}\right)\td{1}_p + A\td{e}_z\otimes\td{e}_z$$
On note que si $B=0$, le chargement macroscopique est purement déviatorique tandis que si $A=0$, on a un chargement purement hydrostatique dans le plan.
\end{solution}

Ce champ de vitesse est \textbf{cinématiquement admissible} avec un tenseur des déformations $\td{D} = D_{xx} (\td{e}_x \otimes \td{e}_x + \td{e}_y \otimes \td{e}_y) + D_{zz} \td{e}_z \otimes \td{e}_z $ et \textbf{plastiquement admissible}, \textit{i.e.}, est compatible avec la plasticité de von Mises qui impose un écoulement incompressible. De plus, il est possible de montrer que $\td{v}(\partial V) = \td{D} \cdot \td{x}$, avec $\td{x}$ le vecteur position, ce qui correspond à des conditions aux limites de type \textbf{homogène au bords en déformation}. Ces trois conditions justifient théoriquement l'utilisation de l'analyse limite (Eq.~\ref{eq2}). %Dans la suite, on fait l'hypothèse que $A \geq 0$ et $B \geq 0$.\\


\question Evaluer le travail des efforts extérieurs $\tSigma \cdot \td{D}$ en fonction de $A$ et $B$. On pourra poser $(\Sigma_{xx}+\Sigma_{yy})/2= \Sigma_\ell$.
\begin{solution}
Le travail s'écrit ici:
\begin{eqnarray*}
\tSigma \cdot \td{D} &= \Sigma_{xx}D_{xx}+\Sigma_{yy}D_{xx}+\Sigma_{zz}D_{zz}\\
&= \left(\dfrac{B}{b^2}- \dfrac{A}{2}\right)(\Sigma_{xx}+\Sigma_{yy})+\Sigma_{zz}A \\
&= \dfrac{B}{b^2}2\Sigma_\ell + (\Sigma_{zz}-\Sigma_\ell)A
\end{eqnarray*} en notant $(\Sigma_{xx}+\Sigma_{yy})/2= \Sigma_\ell$.
\end{solution}

\question Donner l'expression de la dissipation plastique $\Pi$ (Eq.~\ref{eq3}) sous forme d'une intégrale.
\begin{solution}
On a ici:
\begin{align*}\Pi(\td{D}) &= \frac{1}{V} \int_V \sigma_0 d_{eq}(\td{v}) \,dV \\
&= \frac{1}{\pi b^2 L} \left(\int_0^L dz\right)\left(\int_0^{2\pi} d\theta\right)\int_a^b \sigma_0 d_{eq}(v) rdr\\
&= \frac{2}{b^2} \int_a^b \sigma_0 d_{eq}(v) rdr
\end{align*}
où:
$$d_{eq} = \sqrt{\dfrac{2}{3}}\sqrt{d_{rr}^2+d_{\theta\theta}^2+d_{zz}^2} = \sqrt{\dfrac{4}{3}\dfrac{B^2}{r^4}+A^2}$$
\end{solution}
\end{questions}

\section{Estimations}
Avant d'évaluer cette intégrale dans le cas général, nous considérons deux cas particuliers, à savoir $(B = 0, A \neq 0)$ et  $(B \neq 0, A = 0)$.\\

\begin{questions}
\question En calculant la dissipation plastique $\Pi$ et explicitant l'Eq.~\ref{eq2} dans chacun de ces cas, déterminer les estimations du critère de plasticité associées.
\begin{solution}
Dans le cas $B=0$, on a $d_{eq}=|A|$ ce qui donne:
$$ \Pi(\td{D}) = \frac{2}{b^2} \int_a^b \sigma_0 |A| rdr = \sigma_0 |A|(1-f)$$
Compte-tenu du fait que $\tSigma \cdot \td{D} = (\Sigma_{zz}-\Sigma_\ell)A$, on déduit de \eqref{eq2} que:
\begin{align*}
\Phi(\tSigma) \leq 0 &\Longrightarrow (\Sigma_{zz}-\Sigma_\ell)A \leq \sigma_0 (1-f)|A| \quad \forall A\\
&\Longrightarrow |\Sigma_{zz}-\Sigma_\ell| \leq \sigma_0(1-f)
\end{align*}
On en déduit que le critère de plasticité macroscopique est nécessairement \textbf{contenu} dans le critère défini par $|\Sigma_{zz}-\Sigma_\ell| \leq \sigma_0(1-f)$. Il s'agit du critère de la matrice diminué d'un facteur $(1-f)$.\\

De la même façon, pour le cas $A=0$, on a $d_{eq} = \dfrac{2}{\sqrt{3}}\dfrac{|B|}{r^2}$ de sorte que:
$$ \Pi(\td{D}) = \frac{2}{b^2} \int_a^b  \dfrac{2}{\sqrt{3}}\sigma_0 \dfrac{|B|}{r} dr = \dfrac{4}{\sqrt{3}}\sigma_0 \dfrac{\ln(b/a)}{b^2}|B|$$
Compte-tenu du fait que $\tSigma \cdot \td{D} = 2\Sigma_\ell B/b^2$, on déduit de \eqref{eq2} que:
\begin{align*}
\Phi(\tSigma) \leq 0 &\Longrightarrow \Sigma_\ell B \leq \dfrac{2}{\sqrt{3}}\sigma_0 \ln(b/a)|B| \quad \forall B\\
&\Longrightarrow |\Sigma_\ell| \leq \dfrac{2}{\sqrt{3}}\sigma_0 \ln(b/a)
\end{align*}
On en déduit que le critère de plasticité macroscopique est nécessairement \textbf{contenu} dans le critère défini par $|\Sigma_\ell| \leq -\dfrac{1}{\sqrt{3}}\sigma_0 \ln(f)$. On notera qu'il s'agit de la pression limite d'un cylindre sous pression.

En conclusion, le critère macroscopique est contenu dans l'intersection de ces deux critères. Il s'agit là d'une approche cinématique par l'extérieure de l'analyse limite.
\end{solution}

\question Tracer l'estimation précédente du critère de plasticité dans le plan $( \Sigma_\ell, \Sigma_{zz} - \Sigma_\ell )$\\
\begin{solution}

\begin{R_figure}
\includegraphics[width=0.6\textwidth]{Gurson_cylinder_approx}
\captionof{figure}{Estimation du critère macroscopique pour différentes valeurs de $f$. Le critère est exact est nécessairement situé à l'intérieur de cette estimation.}
\end{R_figure}
\end{solution}
\end{questions}

\section{Expression dans le cas général}
Dans le cas général, il est possible de montrer que:
  \begin{align}
    \Pi(\td{D}) &= \sigma_0 A \left(\lambda M(\lambda)+N(\lambda)\right) \ \ \ \ \ \mathrm{avec} \ \ \ \ \ \lambda = \frac{2B}{\sqrt{3}Ab^2} \\
    M(\lambda) &=\mathrm{asinh}{\left( \frac{\lambda}{f} \right)} -\mathrm{asinh}{\left( \lambda \right)}\\
    N(\lambda) &= \sqrt{1 + \lambda^2} - \sqrt{f^2 + \lambda^2}
  \end{align}
On note en particulier que $N'(\lambda)= -\lambda M'(\lambda)$.

\begin{questions}
\question En explicitant l'Eq.~\eqref{eq:plastic-boundary}, montrer que les états de contrainte $\tSigma^*$ à la frontière du domaine sont donnés par:
  \begin{equation}
    \frac{\Sigma_{zz}^* - \Sigma_\ell^*}{\sigma_0} = N(\lambda)
\end{equation}  
    \begin{equation}
    \frac{\Sigma_\ell^*}{\sigma_0} = \frac{M(\lambda)}{\sqrt{3}}
\end{equation}  
\begin{solution}
$\td{D}$ est à présent paramétré par $A$ et $\lambda$. Ainsi, le travail des efforts extérieurs devient:
$$\tSigma \cdot \td{D} = \sqrt{3}A\lambda\Sigma_\ell + (\Sigma_{zz}-\Sigma_\ell)A.$$
Pour les états de contrainte situés à la frontière du domaine, on a $\tSigma^* \cdot \td{D} = \Pi(\td{D})$ ce qui donne:
$$  \sqrt{3}\lambda\Sigma_\ell^* + \Sigma_{zz}^*-\Sigma_\ell^* = \sigma_0 \left(\lambda M(\lambda)+N(\lambda)\right)$$
En dérivant par rapport à $\lambda$ à gauche et à droite, on obtient:
$$ \sqrt{3}\Sigma_\ell^* = \sigma_0 M(\lambda)+ \sigma_0(\lambda M'(\lambda)+N'(\lambda))$$
Comme le dernier terme s'annule par construction, on a bien:
$$\frac{\Sigma_\ell^*}{\sigma_0} = \frac{M(\lambda)}{\sqrt{3}}$$
D'où on déduit de l'expression précédente que:
$$\Sigma_{zz}^*-\Sigma_\ell^* = \sigma_0 N(\lambda)$$
\end{solution}
%donner les expressions de $\Sigma_{xx}$ et $\Sigma_{zz}$ en fonction de $D_{xx}$ et $D_{zz}$.\\
  
\question En combinant les expressions précédentes, montrer que:
  \begin{equation}
   \Phi\left( \Sigma     \right)  = \left(  \frac{\Sigma_{zz} - \Sigma_\ell}{\sigma_0} \right)^2 + 2f \cosh{\left(  \sqrt{3} \frac{\Sigma_\ell}{\sigma_0} \right)} - (1 + f^2) = 0
\label{eq4}
  \end{equation}  
\begin{solution}
On rappelle que $\cosh(a-b)=\cosh(a)\cosh(b)-\sinh(a)\sinh(b)$ et que $\cosh(\mathrm{asinh}(x)) = \sqrt{1+x^2}$. Ainsi, on a:
\begin{align*}
\cosh{\left(  \sqrt{3} \frac{\Sigma_\ell}{\sigma_0} \right)} &= \cosh{\left[\mathrm{asinh}{\left( \frac{\lambda}{f} \right)}\right]}\cosh[\mathrm{asinh}{(\lambda)}] - \sinh{\left[\mathrm{asinh}{\left( \frac{\lambda}{f} \right)}\right]}\sinh[\mathrm{asinh}{(\lambda)}]\\
&= \sqrt{\left(\frac{\lambda}{f} \right)^2+1}\sqrt{\lambda^2+1} - \frac{\lambda^2}{f}
\end{align*}
En remarquant que:
$$\left(  \frac{\Sigma_{zz} - \Sigma_\ell}{\sigma_0} \right)^2 = N(\lambda)^2 = 1+\lambda^2+f^2+\lambda^2 -2\sqrt{1+\lambda^2}\sqrt{f^2+\lambda^2}$$
on vérifie bien l'expression demandée.
\end{solution}
\end{questions}

\section{Etude du critère}
L'Eq.~\eqref{eq4} correspond au critère de plasticité pour matériau poreux recherché, dont il convient de vérifier la pertinence en étudiant les cas limites.\\

\begin{questions}
\question Que devient l'Eq.~\eqref{eq4} dans la limite $f \to 0$ ? Et dans la limite $f \to 1$ ?
\begin{solution}
Lorsque $f\to 0$, il n'y a plus de pore et on retombe sur le critère de von Mises de la matrice. \`A la limite $f\to 1$, compte-tenu du fait que $\cosh(x)\geq 1$, on constate qu'il n'est plus possible de vérifier le critère sauf si $\tSigma=0$.
Cela est cohérent avec les deux estimations précédentes qui dégénèrent lorsque $f\to 1$.
\end{solution}

La figure \ref{fig:Gurson_crit} trace le critère dans le plan $(\Sigma_\ell, \Sigma_{zz} - \Sigma_\ell)$ pour différentes valeurs de porosité.

\begin{figure}
\begin{center}
\includegraphics[width=0.6\textwidth]{Gurson_cylinder}
\end{center}
\caption{Critère de Gurson dans le plan $(\Sigma_\ell, \Sigma_{zz} - \Sigma_\ell)$ pour différentes valeurs de porosité. En noir pointillés: le critère de von Mises de la matrice. En couleurs pointillées: les estimations de la section 2.}
\label{fig:Gurson_crit}
\end{figure}

Le critère de plasticité obtenu doit être accompagné d'une loi d'écoulement, de la même manière que pour la plasticité de von Mises (Eq.~\ref{eq200}). Hill \cite{hill}, puis Gurson dans le cas des matériaux poreux \cite{gurson}, ont montré grâce au principe de dissipation maximale que la loi de normalité est conservée au niveau macroscopique lors du changement d'échelle. On a donc:

  \begin{equation}
     \td{D} = \dot{\Lambda} \frac{\partial \phi(\tSigma)}{\partial \tSigma},  \quad \dot{\Lambda}\geq 0
  \end{equation}
où $\dot{\Lambda}$ est le multiplicateur plastique de la loi de plasticité macroscopique.

\question Déterminer $D_{xx}$, $D_{zz}$ en fonction de $\Sigma_\ell$, $\Sigma_{zz}$.
 \begin{solution}
 On a:
\begin{align*}
\frac{\partial \phi(\tSigma)}{\partial \tSigma} &= \frac{\partial \phi(\tSigma)}{\partial \Sigma_\ell}\frac{\partial \Sigma_\ell}{\partial \tSigma} + \frac{\partial \phi(\tSigma)}{\partial \Sigma_{zz}}\frac{\partial \Sigma_{zz}}{\partial \tSigma}\\
 &= \frac{\partial \phi(\tSigma)}{\partial \Sigma_\ell}\dfrac{1}{2}\td{1}_p + \frac{\partial \phi(\tSigma)}{\partial \Sigma_{zz}}\td{e}_z\otimes\td{e}_z
\end{align*}
On a donc:
 $$D_{zz} = \dot{\Lambda}\dfrac{\partial \phi(\tSigma)}{\partial \Sigma_{zz}} = 2\dot{\Lambda}\dfrac{\Sigma_{zz}-\Sigma_\ell}{\sigma_0^2}$$
et
$$D_{xx} = \dfrac{\dot{\Lambda}}{2}\dfrac{\partial \phi(\tSigma)}{\partial \Sigma_\ell} = -\dot{\Lambda}\dfrac{\Sigma_{zz}-\Sigma_\ell}{\sigma_0^2} +  \dot{\Lambda}\dfrac{\sqrt{3}f}{\sigma_0} \sinh{\left(  \sqrt{3} \frac{\Sigma_\ell}{\sigma_0} \right)} $$
 \end{solution} 


 \question Dans le cadre de la modélisation de la rupture ductile, l'évolution de la porosité $f$ est particulièrement d'intérêt. Montrer que $\dot{f} = (1 - f) \mathrm{tr} (\td{D})$. Donner son expression en fonction de $\Sigma_\ell,\Sigma_{zz}$ et $D_{zz}$.
\begin{solution}
La porosité est donnée par $f=V_f/V$ où $V_f$ est le volume des pores et $V=V_f+V_m$ le volume total. La matrice étant incompressible, on a $\dot{V}_m=0$ et donc $\dot{V}=\dot{V}_f$. Ainsi:
$$\dot{f} = \dfrac{\dot{V}_f}{V}-\dfrac{V_f\dot{V}}{V^2} = \dfrac{\dot{V}}{V}\left(1-\dfrac{V_f}{V}\right)$$
Enfin, par définition du taux de déformation, $\mathrm{tr}( \td{D}) =  \dfrac{\dot{V}}{V}$ de sorte que:
$$\dot{f}=(1-f)\mathrm{tr}(\td{D}) = (1-f)(2D_{xx}+D_{zz})$$

Grâce aux relations précédentes, on a:
$$\dot{f} = (1-f)\dot{\Lambda}\dfrac{2\sqrt{3}f}{\sigma_0} \sinh{\left(  \sqrt{3} \frac{\Sigma_\ell}{\sigma_0} \right)} $$
De plus:
$$\dot{\Lambda}= \dfrac{D_{zz}\sigma_0^2}{2(\Sigma_{zz}-\Sigma_\ell)}$$
soit:
$$\dot{f} = (1-f)f\dfrac{\sqrt{3}\sigma_0}{\Sigma_{zz}-\Sigma_\ell}D_{zz}$$
\end{solution}
\end{questions}

\section{Extension aux matériaux anisotropes}
Le critère de plasticité pour matériaux poreux obtenu est pertinent dans le cas des matériaux isotropes. Le matériau caractérisé dans les travaux pratiques (alliage X100) est orthotrope et peut être modélisé, en l'absence de porosité, par la plasticité de Hill:
\begin{equation}
  \phi\left( \sigma   \right) =  \sigma_{eq}^H - \sigma_0 \leq 0 \quad \text{avec } \sigma_{eq}^H = \sqrt{ \td{\sigma} \cdot \tq{A} \cdot \td{\sigma}}
  \label{eq100}
\end{equation}
Le tenseur $\tq{A}$ contient les informations concernant l'anisotropie du matériau. En notation de Mandel, ces tenseurs sont représentés par les matrices suivantes dans les axes d'orthotropies:
\begin{equation}
  \tsigma = \begin{pmatrix}
    \sigma_{11} \\
    \sigma_{22} \\
    \sigma_{33} \\
    \sqrt{2}\sigma_{12} \\
    \sqrt{2}\sigma_{13} \\
    \sqrt{2}\sigma_{23} \\
  \end{pmatrix} \hspace{2cm}
  \tq{A} = \begin{pmatrix}
    F + H & -F & -H & 0 & 0 & 0 \\
    -F & G+ F & -G & 0 & 0 & 0 \\
    -H & -G & G+H & 0 & 0 & 0 \\
    0 & 0 & 0 & L & 0 & 0 \\
    0 & 0 & 0 & 0 & M & 0 \\
    0 & 0 & 0 & 0 & 0 & N \\
    \end{pmatrix}
\end{equation}
Pour $F = G = H = 1/2$ et $L = M = N = 3/2$, le critère de Hill correspond au critère de von Mises.
% On considère ici le cas isotrope transverse d'axe $\td{e}_z$. Dans ce cas, $G=H$ et $L=G+2F$.\\
%La loi d'écoulement plastique associée permet d'obtenir le taux de déformation plastique:
%\begin{equation}
%  \td{d} = d_{eq} \frac{\tq{A}}{\sigma_0}  \hspace{2cm} d_{eq} = \sqrt{\td{d} \cdot \tq{B} \cdot \td{d} }
%  \label{eq10}
%\end{equation}
%où les paramètres du tenseur $\tq{B}$ sont reliés à ceux de $\tq{A}$ par les relations donnés en Annexe.\\
%\begin{itemize}
%  \question Quels paramètres est-il possible d'obtenir à partir des résultats expérimentaux ?\\
%\end{itemize}

Il est possible de conduire la même démarche d'analyse limite que dans le cas isotrope pour prendre en compte l'effet de la porosité, ce qui conduit au critère de plasticité suivant \cite{benzerga}:
  \begin{equation}
   \Phi\left( \Sigma     \right)  = \left(  \frac{\Sigma_{eq}^H}{\sigma_0^2} \right)^2 + 2f \cosh{\left(  \frac{\sqrt{3}}{h_m} \frac{\Sigma_\ell}{\sigma_0} \right)} - (1 + f^2) = 0
\label{eq5}
  \end{equation} 

avec $h_m = \sqrt{\dfrac{3}{2}}\sqrt{\dfrac{H+G}{4(FG+FH+GH)} + \dfrac{1}{2L}}$.
\begin{questions}
\question Dans quelle situation particulière de symétries matérielles est-il raisonnable de considérer le même champ de vitesse qu'en \eqref{eq:champ-vitesse} ?
\begin{solution}
Le champ précédent suppose une invariance par rotation autour de l'axe $z$. Ce champ peut donc être utilisé dans le cas particulier de l'isotropie transverse d'axe $z$.
\end{solution}
    \question \`A partir de l'expression précédente, commenter l'effet de l'anisotropie sur la plasticité des matériaux poreux.
\begin{solution}
On constate que l'anisotropie impacte plusieurs aspects du critère. Premièrement, l'anisotropie impacte la définition de la norme équivalente $\Sigma_{eq}^H$. De plus, si l'on trace le critère dans le plan $(\Sigma_\ell,\Sigma_{eq}^H)$, l'anisotropie impacte, via le facteur $h_m$ la limite suivant l'axe $\Sigma_\ell$. Par conséquent, ce même facteur aura également un effet sur l'évolution de la porosité puisque, si l'on reprend les mêmes calculs que précédemment pour la règle de normalité, on aura notamment:
$$\mathrm{tr}(\td{D}) = \dot{\Lambda}\dfrac{2\sqrt{3}f}{h_m\sigma_0} \sinh{\left(  \sqrt{3} \frac{\Sigma_\ell}{h_m\sigma_0} \right)} = \dfrac{\dot{f}}{1-f}$$
On voit donc que l'effet de l'anisotropie impacte également la loi d'évolution de la porosité.
\end{solution}
\end{questions}

\begin{thebibliography}{9}
\bibitem{hill}
Hill, R. (1967). The essential structure of constitutive laws for metal composites and polycrystals. Journal of the Mechanics and Physics of Solids, 15(2), 79-95.

\bibitem{gurson}
Gurson, A.L. (1975). Plastic flow and fracture behaviour of ductile materials incorporating void nucleation, growth and interaction. PhD thesis. Brown University, Providence, USA.
\bibitem{benzerga}
Benzerga, A. A.,   \& Besson, J. (2001). Plastic potentials for anisotropic porous solids. European Journal of Mechanics-A/Solids, 20(3), 397-434.


\end{thebibliography}
\end{document}
